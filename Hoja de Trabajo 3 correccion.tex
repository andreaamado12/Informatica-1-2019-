\documentclass{article}
\usepackage{fancyhdr} % Required for custom headers
\usepackage{lastpage} % Required to determine the last page for the footer
\usepackage{extramarks} % Required for headers and footers
\usepackage[usenames,dvipsnames]{color} % Required for custom colors
\usepackage{graphicx} % Required to insert images
\usepackage{listings} % Required for insertion of code
\usepackage{courier} % Required for the courier font
\usepackage{multirow}
\usepackage{hyperref}
\usepackage{amsmath}
\usepackage{amssymb}
\usepackage[utf8]{inputenc}

% Margins
\topmargin=-0.45in
\evensidemargin=0in
\oddsidemargin=0in
\textwidth=6.5in
\textheight=9.0in
\headsep=0.25in

\linespread{1.1} % Line spacing

%----------------------------------------------------------------------------------------
%	CODE INCLUSION CONFIGURATION
%----------------------------------------------------------------------------------------

\definecolor{MyDarkGreen}{rgb}{0.0,0.4,0.0} % This is the color used for comments
\lstloadlanguages{c} % Load Perl syntax for listings, for a list of other languages supported see: ftp://ftp.tex.ac.uk/tex-archive/macros/latex/contrib/listings/listings.pdf
\lstset{language=[sharp]c, % Use Perl in this example
        frame=single, % Single frame around code
        basicstyle=\small\ttfamily, % Use small true type font
        keywordstyle=[1]\color{Blue}\bf, % Perl functions bold and blue
        keywordstyle=[2]\color{Purple}, % Perl function arguments purple
        keywordstyle=[3]\color{Blue}\underbar, % Custom functions underlined and blue
        identifierstyle=, % Nothing special about identifiers                                         
        commentstyle=\usefont{T1}{pcr}{m}{sl}\color{MyDarkGreen}\small, % Comments small dark green courier font
        stringstyle=\color{Purple}, % Strings are purple
        showstringspaces=false, % Don't put marks in string spaces
        tabsize=5, % 5 spaces per tab
        %
        % Put standard Perl functions not included in the default language here
        morekeywords={rand},
        %
        % Put Perl function parameters here
        morekeywords=[2]{on, off, interp},
        %
        % Put user defined functions here
        morekeywords=[3]{test},
       	%
        morecomment=[l][\color{Blue}]{...}, % Line continuation (...) like blue comment
        numbers=left, % Line numbers on left
        firstnumber=1, % Line numbers start with line 1
        numberstyle=\tiny\color{Blue}, % Line numbers are blue and small
        stepnumber=5 % Line numbers go in steps of 5
}

\newcommand{\horrule}[1]{\rule{\linewidth}{#1}}

% Creates a new command to include a perl script, the first parameter is the filename of the script (without .pl), the second parameter is the caption
\newcommand{\perlscript}[2]{
\begin{itemize}
\item[]\lstinputlisting[caption=#2,label=#1]{#1.cs}
\end{itemize}
}

\begin{document}



\begin{center}
        \horrule{0.5pt}
        \huge{Hoja de trabajo \#3} \\
        \large{Andrea Amado}\\
        \large{Byron Terre}\\
        \large{Fernando de Tezanos}\\
        \large{13 de Agosto, 2019} \\
        \horrule{1pt}
\end{center}
\section*{Ejercicio \#1}
Utilizando la definicion de suma ($\oplus$) para los numeros naturales unarios, llevar
a cabo la suma entre tres [$s(s(s(0)))$] y cuatro [$s(s(s(s(0))))$].\\ 
\[
        s(s(s(0)))\oplus s(s(s(s(0))))
\]\\
        a=s(x)\\
        x=s(s(0))\\
        b=s(s(s(s(0))))
 \[
        s[s(s(0))\oplus b]
\] \\
        a=s(x)\\
        x=s(0)\\
        b=s(s(s(s(0))))
\[
        s[s(s(0)\oplus b)]
\]\\
        a=s(x)\\
        x=0\\
        b=s(s(s(s(0))))
\[
        s[s(s(0\oplus b))]
\]\\
0+b=b
\[
        s[(s(s(b))]
\]
\[
        [s(s(s(s(s(s(s(0)))))))]
\]
\[
        =7
\]
\section*{Ejercicio \#2}
Definir inductivamente una funci\'on para multiplicar ($\otimes$) numeros naturales unarios.
\[
        a\otimes b := \left\{
        \begin{array}{l l}
            0 & \mbox{si } a=o \\
            0 & \mbox{si } b=o \\
            a & \mbox{si } b=1 \\
            b & \mbox{si } a=1 \\
            b\oplus(x\otimes b) & \mbox{si } a=\sigma(x) \\
        \end{array}
        \right.
    \]

\section*{Ejercicio \#3}
Verifique que su definici\'on de multiplicaci\'on es correcta multiplicando los siguientes valores:
\begin{itemize}
        \item{$\sigma(\sigma(\sigma(0)))\otimes 0$}\\
\[
       \sigma(\sigma(\sigma(0)))\otimes 0\\
\] 
        a=$\sigma(\sigma(\sigma(0)))$\\
\[
       a\otimes 0=0\\
\] 
        
        \item{$\sigma(\sigma(\sigma(0)))\otimes \sigma(0)$}
\[
       \sigma(\sigma(\sigma(0)))\otimes \sigma(0)\\
\] 
        a=$\sigma(x)$\\
        x=$\sigma(\sigma(0))$\\
        b=$\sigma(0)$
\[
       b\oplus(x\otimes b)\\
\]
\[
       =b\oplus(\sigma(\sigma(0))\otimes b)\\
\]
        a=$\sigma(x)$\\
        x=$\sigma(0)$
\[
       =b\oplus b\oplus(\sigma(0)\otimes b)\\
\]
        a=$\sigma(x)$\\
        x=0
\[
       =b\oplus b\oplus b\oplus(0\otimes b)\\
\]
 0 $\otimes$ b=0
\[
       =b\oplus b\oplus b\oplus(0)\\
\]
\[
       =b\oplus b\oplus b\\
\]
\[
       =\sigma(0)\oplus \sigma(0)\oplus \sigma(0)\\
\]
\[
       =\sigma( \sigma( \sigma(0)))\\
\]
\[
       =3\\
\]
        \item{$\sigma(\sigma(\sigma(0)))\otimes \sigma(\sigma(0))$}
        \[
       \sigma(\sigma(\sigma(0)))\otimes \sigma(\sigma(0))\\
\] 
        a=$\sigma(x)$\\
        x=$\sigma(\sigma(0))$\\
        b=$\sigma\sigma((0))$
\[
       b\oplus(x\otimes b)\\
\]
\[
       =b\oplus(\sigma(\sigma(0))\otimes b)\\
\]
        a=$\sigma(x)$\\
        x=$\sigma(0)$
\[
       =b\oplus b\oplus(\sigma(0)\otimes b)\\
\]
        a=$\sigma(x)$\\
        x=0
\[
       =b\oplus b\oplus b\oplus(0\otimes b)\\
\]
 0 $\otimes$ b=0
\[
       =b\oplus b\oplus b\oplus(0)\\
\]
\[
       =b\oplus b\oplus b\\
\]
\[
       =\sigma(\sigma(0))\oplus \sigma(\sigma(0))\oplus \sigma(\sigma(0))\\
\]
\[
       =\sigma( \sigma( \sigma(\sigma( \sigma( \sigma(0))))))\\
\]
\[
       =6\\
\]
\end{itemize}
\section*{Ejercicio \#4}
Demostrar utilizando inducci\'on:
\begin{itemize}
        \item{$a\oplus \sigma(\sigma(0))=\sigma(\sigma(a))$}\\
        \large{Caso Base:}
        a=0
\[
       0+\sigma(\sigma(0))=\sigma(\sigma(0))\\
\]
\[
       \sigma(\sigma(0))=\sigma(\sigma(0))\\
\]
\large{Caso Inductivo:}
a=$\sigma(x)$\\
Hipótesis Inductiva: 
x+$\sigma(\sigma(0))=\sigma(\sigma(x))$
\[
       \sigma(x)\oplus\sigma(\sigma(0))=\sigma(\sigma(\sigma(x)))\\
\]
\[
       \sigma(x\oplus\sigma(\sigma(0)))=\sigma(\sigma(\sigma(x)))\\
\]
\[
       \sigma(\sigma(\sigma(0))\oplus x)=\sigma(\sigma(\sigma(x)))\\
\]
\[
       \sigma(\sigma(\sigma(0)\oplus x))=\sigma(\sigma(\sigma(x)))\\
\]
\[
       \sigma(\sigma(\sigma(0\oplus x)))=\sigma(\sigma(\sigma(x)))\\
\]
\[
       \sigma(\sigma(\sigma(x)))=\sigma(\sigma(\sigma(x)))\\
\]
\\\\
        \item{$a \otimes b = b \otimes a$}\\
\large{Caso Base:}
a=0
\[
       0 \otimes b = b \otimes 0\\
\]
\[
       0 = 0\\
\]
\large{Caso Inductivo:}
a=$\sigma(x)$\\
Hipótesis Inductiva:
$x \otimes b = b \otimes x$
\[
      \sigma(x) \otimes b = b \otimes \sigma(x)\\
\]
$b=\sigma(i)$
\[
     b\oplus (x \otimes b) = b \otimes \sigma(x)\\
\]
\[
     \sigma(i)\oplus (x \otimes \sigma(i)) = \sigma(i) \otimes \sigma(x)\\
\]
\[
     \sigma(i)\oplus (x \otimes \sigma(i)) = \sigma(x)\oplus(i \otimes\sigma(x)) \\
\]
\[
     \sigma(i\oplus (x \otimes \sigma(i))) = \sigma(x\oplus(i \otimes\sigma(x))) \\
\]
\[
     \sigma(i\oplus (x \otimes b)) = \sigma(x\oplus(i \otimes a)) \\
\]
Por Hipótesis inductiva:\\
$x \otimes b=b\otimes x$\\
$i \otimes a=a\otimes i$
\[
     \sigma(i\oplus (b \otimes x)) = \sigma(x\oplus(a \otimes i)) \\
\]
\[
     \sigma(i\oplus (\sigma(i) \otimes x)) = \sigma(x\oplus(\sigma(x) \otimes i)) \\
\]
\[
     \sigma(i\oplus x\oplus (i \otimes x)) = \sigma(x\oplus i \oplus (x \otimes i)) \\
\]
Demostrar que $i\oplus x=x\oplus i$\\
para $i=\sigma (j)$\\
Hipótesis inductiva: $j\oplus x=x\oplus j$\\
$\sigma(j)\oplus x=x\oplus \sigma(j)$\\
$\sigma(j\oplus x)=x\oplus \sigma(j)$\\
$\sigma(x\oplus j)=x\oplus \sigma(j)$\\
$\sigma(x)\oplus j=x\oplus \sigma(j)$\\
Demostrar para el sucesor:\\
Hipótesis inductiva: $\sigma(x)\oplus j=x\oplus \sigma(j)$\\
$\sigma(\sigma(x))\oplus j=\sigma(x)\oplus \sigma(j)$\\
$\sigma(\sigma(x)\oplus j)=\sigma(x\oplus \sigma(j))$\\
Por Hipótesis inductiva:\\
$\sigma(\sigma(x)\oplus j)=\sigma(\sigma(x)\oplus j)$\\
Por lo tanto $i\oplus x=x\oplus i$
\[
     \sigma(x\oplus i\oplus (i \otimes x)) = \sigma(x\oplus i \oplus (x \otimes i)) \\
\]
Por Hipótesis inductiva:\\
$x\otimes i=i\otimes x$
\[
     \sigma(x\oplus i\oplus (x \otimes i)) = \sigma(x\oplus i \oplus (x \otimes i)) \\
\]
k=$x\oplus i\oplus (x \otimes i)$
\[
     \sigma(k) = \sigma(k) \\
\]
\\\\
        \item{$a \otimes (b \otimes c)=(a\otimes b)\otimes c$}\\
\large{Caso Base:}
a=0
\[
       0 \otimes (b \otimes c)=(0\otimes b)\otimes c
\]
\[
       0 \otimes (b \otimes c)=0\otimes c
\]
\[
       0=0 
\]
\large{Caso Inductivo:}
a=$\sigma(x)$\\
Hipótesis Inductiva:
$x \otimes (b \otimes c)=(x\otimes b)\otimes c$
\[
       \sigma(x) \otimes (b \otimes c)=(\sigma(x)\otimes b)\otimes c
\]
\[
       (b\otimes c)\oplus(x \otimes (b \otimes c))=(\sigma(x)\otimes b)\otimes c
\]
\[
       (b\otimes c)\oplus(x \otimes (b \otimes c))= (b\oplus (x\otimes b))\otimes c
\]
Distribuir la multiplicación $b\oplus (x\otimes b))\otimes c$ (ver siguiente demostración $(a\oplus b)\otimes c = (a\otimes c) \oplus (b \otimes c)$):
\[
       (b\otimes c)\oplus(x \otimes (b \otimes c))= (b\otimes c) \oplus ((x\otimes b)\otimes c)
\]
Aplicar propedad conmutativa de la suma, explicada en la demostración anterior
\[
       (b\otimes c)\oplus[(x \otimes (b \otimes c))= ((x\otimes b)\otimes c)]\oplus (b\otimes c)
\]
\begin{center}Asumiendo que $[(x \otimes (b \otimes c))= ((x\otimes b)\otimes c)]$ es verdadero (Hipótesis Inductiva):\end{center}
\[
       (b\otimes c)= (b\otimes c)
\]
\\\\
        \item{$(a\oplus b)\otimes c = (a\otimes c) \oplus (b \otimes c)$}\\
\large{Caso Base:}
a=0
\[
       (0\oplus b)\otimes c = (0\otimes c) \oplus (b \otimes c)
\]
\[
       (b)\otimes c = (0) \oplus (b \otimes c)
\]
\[
       b\otimes c = b \otimes c
\]
\large{Caso Inductivo:}
a=$\sigma(x)$\\
Hipótesis Inductiva:
$(x\oplus b)\otimes c = (x\otimes c) \oplus (b \otimes c)$
\[
    (\sigma(x)\oplus b)\otimes c = (\sigma(x)\otimes c) \oplus (b \otimes c)   
\]
\[
    \sigma(x\oplus b)\otimes c = (\sigma(x)\otimes c) \oplus (b \otimes c)   
\]
\[
    c\oplus ((x\oplus b)\otimes c) = (\sigma(x)\otimes c) \oplus (b \otimes c)   
\]
\[
    c\oplus ((x\oplus b)\otimes c) = (c\oplus(x\otimes c)) \oplus (b \otimes c)   
\]
Aplicar propedad conmutativa de la suma, explicada dos demostraciones atrás
\[
    c\oplus [(x\oplus b)\otimes c) = (x\otimes c)) \oplus (b \otimes c)] \oplus c
\]
\begin{center}Asumiendo que $[(x\oplus b)\otimes c) = (x\otimes c)) \oplus (b \otimes c)]$ es verdadero (Hipótesis Inductiva):\end{center}
\[
   c=c
\]

\end{itemize}
\end{document}